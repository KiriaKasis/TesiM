\usepackage[greek,italian,english]{babel}
\usepackage{amsmath, amsthm, amsopn}
\usepackage{amsfonts}
\usepackage{amssymb}
\usepackage{bm}
\usepackage{lmodern}
%\usepackage{fourier}
%\usepackage[garamonf]{mathdesign}
%\usepackage{fontspec}  

\usepackage[T1]{fontenc}
%\usepackage[left=0.5cm,
%			right=8cm,top=0.5cm,
%			bottom=0.5cm]{geometry}
%\usepackage{mathrsfs}
%\usepackage{bussproofs}
%\usepackage{prftree}
\usepackage{ebproof}
%\usepackage{proof}



\theoremstyle{plain}% default
\newtheorem{thm}{Theorem}[section]
\newtheorem{lem}[thm]{Lemma}
\newtheorem{prop}[thm]{Proposition}
\newtheorem*{cor}{Corollary}
\newtheorem*{KL}{Klein’s Lemma}

\theoremstyle{definition}
\newtheorem{dede}{Definition}[section]
\newtheorem{conj}{Conjecture}[section]
\newtheorem{exmp}{Example}[section]

\theoremstyle{remark}
\newtheorem*{rem}{Remark}
\newtheorem*{note}{Note}
\newtheorem{case}{Case}

\newcommand{\Va}{\mathcal{V}}
\newcommand{\Ve}{\mathcal{E}}
\newcommand{\cp}[1][\cdot,\cdot]{<#1>}
\newcommand{\ctr}{\triangleright}
\newcommand{\sbs}[3]{#1\left\lbrace #2 / #3 \right\rbrace}

\newcommand{\uplaOp}[3]{#1_{1}#3\dots#3#1_{#2}}
\newcommand{\upla}[2]{\uplaOp{#1}{#2}{,}}
\newcommand{\xnContext}[3]
	{#2_1   \in #1_1 
						, \dots, 
	#2_{#3} \in #1_{#3}{(\uplaOp{#2}{#3-1}{,})}}
	
	
\newcommand{\femph}{\textbf}