\chapter{The Interpretation}
	As I understand it the speciality of a 
	realizzability 	interpretation is in 
	being looser, for example in this theory
	the term $natrec(x, 0, 0)$, where x is a 
	free variable, belongs to every type despite 
	not even being well formed in a type theory context \footnote{Ho risolto il  problema!! tutte le \emph{astrazioni-funzioni} sono delle $\lambda$ nella realizzabilità tanto solanto i termini di arietà 0 possono essere tipati in \femph{MTT}}.



\section{Copycatting: Let's get vaguely serious}
	\subsection{Starting definitions}
		We are going to give a different interpretation 
		of the same language used in the Minimalist 
		Foundation. Una delle differenza sarà che definiremo
		un'untyped relazione di riducibilità invece 
		del giudizio di uguanza tipata; then using the distinction in \femph{canonical} and \femph{non-canonical} expression we will define the set of \femph{normal} expressions (le espressioni normali in mtt sono normali anche qui)
		Let's give a simple grammar for our language~(in this interpretation we don't actually diffenrentiate between types and elements): let $\Va$ be an infinite set of variables (generally denoted with $x,y,z,w$ and with \footnote{pensando di nuovo a come trattare il concetto delle \emph{astrazioni} in \textsl{proglöf} credo che per ora la cosa migliore sia dimenticarsene e non usarle}		
\begin{dede}
	The set of expression $\Ve$ will be inductively genererated from the set of variables $\Va$ as closed under the following construct:
	\begin{gather}
	 \lambda x.e \\
	 apply(e_1,e_2) \\
	 0 \\ 
	 succ(e) \\ 
	 natrec(e_1,e_2,e_3) \\ 
	 \cp[e_1,\,e_2] \\
	 El_\Sigma(e_1, e_2) \\
	 N 		\\
	 \Sigma(e_1,e_2)\\
	 \Pi(e_1,e_2)
	\end{gather}
	where $x$ is a variable and each $e_i$ is a previously constructed expression.
	\end{dede}
	\begin{dede}
		An expression is said to be \femph{canonical} if it is in the form $\lambda x.e$, 0, $succ(e)$, $\cp[e_1,e_2]$ 
	\end{dede}
	
	\begin{dede}
		the \femph{contraction} relation ($\ctr$)
		is defined by:
		\begin{equation}
		\begin{split}
		natrec(0, b, c) & \ctr  b \\
		natrec(succ(a), b, c) & \ctr  apply(apply(c,  a), natrec(a, b, c)) \\
		apply(\lambda x.e, d) & \ctr  \sbs{b}{a}{c}\\
		p_0(\cp[a,b]) & \ctr a\\
		p_1(\cp[a,b]) & \ctr b		
		\end{split}
		\end{equation}	
	\end{dede}
	
	\begin{dede}
	Here we give the definition of $\psi$ and $\Psi$ as inference rules:
	\begin{gather*}
	\begin{prooftree}
	\Infer{0}[${}_\RN$]{\Psi(N)}
	\end{prooftree}	\label{def:RN}\\
	%	
	\begin{prooftree}
	\Hypo{\Psi(A)}
	\Hypo{\Psi(B(a)) \left[\psi(A,h,a)\right]}
	\Infer2[${}_\RS$]{\Sigma(A,B)}		
	\end{prooftree}\\
	%
	%
	\begin{prooftree}
	\Hypo{\Psi(A)}
	\Hypo{\Psi(B(a)) \left[\psi(A,h,a)\right]}
	\Infer2[${}_\RP$]{\Pi(A,B)}		
	\end{prooftree}\\
	%
	\begin{prooftree}
	%
	\Hypo{\mathcal{F}(A)}
	\Infer1[${}_\RF$]{\Psi(A)}
	\end{prooftree}
	%
	\begin{prooftree}
	\Hypo{A\rightarrow B}
	\Hypo{\Psi(B)}
	\Infer2[${}_\RAr$]{\Psi(A)}
	\end{prooftree}
	\end{gather*}
	
	
	\begin{gather}
	%
	\begin{prooftree}
	\Hypo{a \rightarrow b}
	\Hypo{\mathcal{N}(b) }
	\Infer2{\psi(N,a)}
	\end{prooftree}	
		%
	\\
	\begin{prooftree}
		\Hypo{c_{1b} \colon \Psi(\Pi(A,B))}
		\Hypo{\psi(B(a), apply(b, a))[\psi(A,a)]}
		\Infer{2}{\psi(\Pi(A,B),c_?,b)}
	\end{prooftree}\\		
		%
	\begin{prooftree}
	\Hypo{c_{1c} \colon \Psi(A)}
	\Hypo{a \rightarrow b}
	\Hypo{\mathcal{F}(b)}
	\Infer3{\psi(A,a)}
	\end{prooftree}	\\		
	%
	\begin{prooftree}
	\Hypo{c_{1d}(A,B,h) \colon \Psi(A)}
	\Hypo{\psi(B,a)}
	\Infer2{\psi(A,a)}
	\end{prooftree}	
	\end{gather}
	
	\begin{gather}
	%
	\begin{prooftree}
	\Hypo{\phantom{\Psi(A)}}
	\Infer1{\Phi(U)}
	\end{prooftree}
	%
	\begin{prooftree}
	\Hypo{\Psi(A)}
	\Infer1{\phi(U,A)}
	\end{prooftree}	
	\end{gather}	
	
	\end{dede}
	
	The normalization theorem will be proved in two steps. In the first step we will prove basic properties of this \emph{Realizzability Interpretation} (such as being normalizing and having other computational properties). In the second step we will give a proof of compatibility between the Minimalist Foundation and the 
	
 	


	