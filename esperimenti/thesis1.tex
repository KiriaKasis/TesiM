\documentclass[12pt,a5paper,draft,oneside]{report}              
%\usepackage[utf8]{inputenc}
\usepackage[T1]{fontenc}
\usepackage[greek,italian,english]{babel}
\usepackage{amsmath}
\usepackage{amsthm}
\usepackage{amsfonts}
\usepackage{amssymb}
\usepackage{bm}
%\usepackage{makeidx}
%\usepackage{graphicx}
%\usepackage{lmodern}
\usepackage{fourier}
%\usepackage[garamond]{mathdesign}
%\usepackage{fontspec}  
%\setmainfont{EB Garamond}[
%  Contextuals={Alternate},
%  Numbers=OldStyle,
%]
%\usepackage{kpfonts}
\usepackage[left=0.5cm,
			right=6cm,top=0.5cm,
			bottom=0.5cm]{geometry}
%\usepackage{mathrsfs}
%\usepackage{bussproofs}
%\usepackage{prftree}
\usepackage{ebproof}
\usepackage{proof}

\author{Alberto Fiori}
\title{Realizzabilità e altre cose interessanti}

\begin{document}

\begin{abstract}
In questa tesi utiliziamo $A_{\bm{\Lambda}\Lambda}$
\end{abstract}

\chapter{ExpQreQQssQionQ QLanguage}
	Following the ideas that I first read in the book "programming in Martin Löf type thery". Basically every expression we write in a mathematical expression has an arity (0,$\alpha\rightarrow\beta$, $\alpha_1\otimes\alpha_2\ldots\otimes\alpha_n$) which work similarly to how types in simply typed lambda calculus work. 
	This simply is a way to ensure that expression are well-formed (even if it doesn't ensure that they are reasonable); we chose to add the $\otimes\ldots\otimes$ constructor to freely chose when to \emph{curry} and \emph{uncurry} function application; we could have added more, but there seem to be no advantage in doing so.
	
	The rule for arity are the expected ones:
	%TODO non so come fare le deduzioni naturali
	
	
	\section{Everithhung has an Arithy} 
		123123123123

\chapter{Test stuff}
	\begin{eqnarray}
		\infer{B}{A & (A \rightarrow B)}	\\
		\infer{B}{A & (A \rightarrow B)}	\\
		\infer={B \land A}{A \land B)}		\\
		\infer[^{(2)}]
     {\neg(\phi \land \psi)}
     {\infer[^{(1)}]
        {\bot}
        {(\neg\phi \lor \neg\psi) & \bot & \bot}
     }
     \\
     \infer[^{(2)}]
     {\neg(\phi \land \psi)}
     {\infer[^{(1)}]
        {\bot}
        {(\neg\phi \lor \neg\psi) & 
        \infer
            {\bot}
            {\phi & 
            \infer[^{(1)}]
            {\neg\phi}{}
            } 
        & \bot}
     }\\     
	\end{eqnarray}
$$
\infer[(\to I)]
     {(A \to B)}
     {
     \infer*{B}{[A]}
     }
\qquad
\infer[(\to E)]
     {B}
     {(A \to B) & A}
$$



The rule 
$\vcenter
{
\infer
    {(A \to B)}
    {
    \infer*{B}{[A] & A & A & A & A & A & A}
    }
}$
is known as $\to$-introduction
\newpage
\[
\begin{prooftree}
\Hypo{ \vdash A }
\Hypo{ \vdash B } \Infer1{ \vdash B, C }
\Infer2{ \vdash A\wedge B, C }
\end{prooftree}
%
\quad \rightsquigarrow \quad
\begin{prooftree}
\Hypo{ \vdash A } \Hypo{ \vdash B }
\Infer2{ \vdash A\wedge B }
\Infer1{ \vdash A\wedge B, C }
\end{prooftree}
\]

\end{document}